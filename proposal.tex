\documentclass[a4paper, 11pt]{article}

\usepackage{kotex} % Comment this out if you are not using Hangul
\usepackage{fullpage}
\usepackage{hyperref}
\usepackage{amsthm}
\usepackage[numbers,sort&compress]{natbib}
\usepackage{indentfirst}

\theoremstyle{definition}
\newtheorem{exercise}{Exercise}

\begin{document}
%%% Header starts
\noindent{\large\textbf{IS-521 Activity Proposal}\hfill
                \textbf{Wi Seong Il}} \\
         {\phantom{} \hfill \textbf{seongil-wi}} \\
         {\phantom{} \hfill Due Date: April 15, 2017} \\
%%% Header ends

\section{Activity Overview}
  \subsection{활동에 대한 설명}
  주어진 Software Sample 코드에 대하여 Secure Coding에 의거하여 직접 취약한 부분을 찾아내고, 이에 따라 scanning solution을 제안 및 제작해 보는 것.
  \subsection{이 활동을 제안하는 이유}
   근간에 취약점 공격은 인프라가 갖추어져 있는 네트워크나, 시스템 보다 Application Level에서의 공격이 주를 이루고 있는 상황이다.\cite{Application}
   일반적인 engineering의 단계는 기본적으로 Requirement Specification, Design, Implement, Verification, Release순으로 이루어 진다. Release이후에 발견되는 보안상 취약점을 고치기 위해 드는 Overhead가 매우 큼을 감안하여, Engineering의 초기 단계부터 보안적 요소를 점검하고, 고쳐나가야 할 필요성을 이 활동을 통해 드러내고자 한다. 특별히, 수동적인 검사활동과 함께 후에 자동적으로 Detect해주는 Solution을 Organize함으로써 보안적인 접근방법을 익혀보는 것을 목표로 한다.
\section{Exercises}

\begin{exercise}
 
  취약한 Software Sample코드를 제시하여 수강자들로 하여금 취약점을 공격하는 활동을 제안한다. 자신이 보고 탐지하기에 가능한 공격들을 모두 찾고, 그에 따라 공격코드나, script를 쓰도록 한다.

\end{exercise}

\begin{exercise}

  취약한 Software Sample코드를 제시하여 수강자들로 하여금 어떠한 부분이 취약한지 찾아서 고치는 활동을 제안한다. 다음 링크\cite{Standard}는 CERT에서 제안하는 C language 시큐어코딩의 standard이다. 이를 참고하여 및 응용하여 주어진 코드상에서 취약점을 정적으로 분석하도록 한다.
  고친부분에는 주석으로 왜 이 부분을 고쳤고, 어떤 공격이 가능한지 기술하도록 한다.

\end{exercise}

\begin{exercise}

  주어진 Sample코드 중에서 취약성으로 인해 고친 부분을 참고하여 취약점을 Detect할 수 있는 Solution을 제시하는 것. (Compiler의 요소들을 활용하는 것도 하나의 방법이 될 것 임) 나온 Output은 기존의 취약한 Software Sample코드를 scanning할 때 본인이 찾았던 취약한 코드들을 자동적으로 Detect할 수 있어야 함. 

\end{exercise}

\section{Expected Solutions}
  \subsection{Milestone}
   \begin{itemize}
      \item Attacking Code
      \item Defense Code(Secure Coding Code)
	  \item Solution
   \end{itemize}
  \subsection{목표}
   \begin{itemize}
      \item Software의 보안상 취약점으로 인하여 어떤 피해가 발생하는지 부분적으로 알고, 문제가 심화될 경우 심각 할 수 있음을 깨닫는 것.
      \item 이에 따라 Engineering의 초기 단계부터 보안적 요소를 가미하는 것이 중요함을 깨닫는 것.
   \end{itemize}

\bibliography{references}
\bibliographystyle{plainnat}

\end{document}
